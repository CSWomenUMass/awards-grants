\documentclass{article}
\usepackage{fullpage}

\begin{document}

\newcommand{\appsection}[2]{
\begin{center}
		\noindent
	\fbox{
		\parbox{\textwidth}{
		  #1
		}
	}
	\fbox{#2 words}		
\end{center}
}
\section*{Description of Gift Usage}

\emph{TIP: It is often helpful to be specific about the components of your event, such as benefits and assessment. For example, while every project is different, try to avoid vague assessment plans such as simply writing, "We will give surveys at the end of the semester."}

\appsection{Describe your activity including its purpose, how you will get people to participate, and the actions they will take as part of your activity.}{300}

CS Women will organize, host, and promote a series of technical workshops designed to build every day skills that students who are introduced to computer science in an academic setting often lack. These skills include (but are not limited to): bash scripting, version control, build tools, graphing/plotting tools, building simulations, LaTeX, and choosing the appropriate language and/or tools for tasks. In February 2015 we distributed a preliminary survey to assess what skills our members would be interested in building. This initiative received an enthusiastic response from our members. As a result of this survey and subsequent discussions, we have already been able to recruit 5 instructors to teach Unix basics, vi/emacs, \LaTeX, and basic bash scripting.

The workshops will be taught by Computer Science graduate students (and some exceptional undergraduates) of all genders. We aspire to have gender parity in our workshop instructors.  Our initial cohort of instructors will have the opportunity to develop a curriculum for a diverse audience; we believe this opportunity presents an attractive supplement or alternative to teaching assistantships for students who want to become teachers.

We intend to promote a collaborative atmosphere where attendees feel comfortable taking risks. Our target level of technological literacy is that of a first-year graduate student. However, we invite all students in the Five College Consortium to attend.

There are several ways in which we will assess our program’s efficacy. We will require students to manage their code on a version control hosting service, such a GitHub. GitHub maintains statistics about user participation. This will allow us to track student engagement over time. Furthermore, some aspects of these workshops will be cumulative. We intend to make our teaching resources publicly available and encourage those who have reached proficiency to lead future workshops.


\appsection{Provide a detailed budget describing how the \$1,000 gift will be used (e.g., our group will spend \$xxx on food and \$xxx on transportation for high school students to attend)}{300}

We intend to launch this workshop series during the Fall 2015 semester. If the Fall semester workshops prove successful, we will continue the workshop series into the Spring 2016 semester. All proposals listed here are for the Fall 2015 semester only.

We expect to run 6 workshop sessions in the Fall 2015 semester and spend approximately \$100 per session (\$600 total) on snacks, coffee, juice, and soft drinks.  This estimate is based on the costs associated with a comparable Python and data analysis workshop series run by the UMass Graduate Women in STEM, which had approximately 35 attendees per session.

Preparing course materials is a time consuming and iterative process. The success of this workshop series depends on the success of the first few workshop sessions. We are relying on our volunteer instructors to create this success and want to make sure that they feel supported and appreciated. To this end, we intend to spend the remaining \$400 on a 2-day weekend orientation for them. This orientation will include a teacher training for interactive, small-group settings. We expect student teachers to have a first draft of their materials ready. They will practice presenting their workshop sessions and experiment with different teaching methodologies. This orientation will also be an opportunity for the teachers to bond with each other. This orientation will take place over a weekend at end of August or early September 2015 at Mt. Holyoke College. We expect to spend \$40/person, or \$240-300 total on food over the weekend. We have received a quote of \$150 to rent the facilities for two weekend days.



\appsection{How does this activity help to promote or sustain the Women in CS/IS/IT group at your university?}{300}
 This activity will promote the visibility of technical women by having gender parity amongst the instructors. We will achieve gender parity by recruiting our initial set of instructors from two overlapping pools: our CS Women student group and our School of Computer Science graduate student population.
 
Parity in the instructors achieves two goals: it increases the visibility of technically strong women while emphasizing the role of male allies. Approximately 20\% of our graduate students and 10\% of our undergraduate majors are women. With over 250 graduate students and over 700 undergraduate majors, this means that male students are the most visible. Promoting the workshop instructors within the department will lend credibility to the women who teach them, emphasizing their expertise in a public forum. Having half the instructors be men will send the message that women do not need to bear the burden of education and outreach alone.

The activities themselves bolster the technical skills of attendees. We also believe that they will promote good work habits that increase the productivity and confidence of attendees. The particular skills we wish to address are those that are most likely to elicit a “read the manual” response from peers. These interactions can be damaging to students, who may feel that there is knowledge they ought to have, but do not, and may conclude that asking technical questions is not culturally acceptable. We will promote a collaborative and supportive environment that encourages students to share technical work habits and help each other in constructive ways.


\appsection{Has this event been held before and if so, were there any successful outcomes?}{300}
The UMass Graduate Women in STEM (GWiS) organization has held a series of introductory Python courses taught by Computer Science graduate students. The GWiS workshops have been capped at 50 participants and have had a waiting list for each of their 4 sessions.  GWiS has held 4 additional workshops on more advanced data analysis, which have had 20-40 attendees.

UMass CS Women conceived of this technical workshop series to address skills gaps in our graduate student population. However, after reaching out to GWiS, we feel that such a workshop series could have a much broader appeal. After a preliminary meeting with the executive board of GWiS and discussion of this seed funding proposal, we determined that a hand off of the Python workshops to UMass CS Women would be mutually beneficial. UMass CS Women has recruited one of the Python instructors for our workshop series.

The School of Computer Science offers several 1-credit undergraduate classes that teach practical computing skills, such as “A Hands-On Introduction to UNIX” and “Programming in C.” These courses are very popular with students. However, they are only open to undergraduate majors. We have recruited one of the UNIX instructors for our workshop series.

\appsection{How will participants benefit?}{300}

Attendees benefit by learning new skills that will greatly increase their productivity and fluency with technology. These are not skills that are typically taught in the classroom, but students are nonetheless expected to possess them, both in industry and in graduate school. We hope that attendees who return for multiple sessions will gain sufficient confidence with the technologies to later lead workshop sessions themselves.

Instructors will benefit from the unique teaching experience this series will provide. They will have the opportunity to assemble a curriculum and teach material in a hands-on environment. Workshop development may count toward a computer science graduate student’s teaching requirement, pending approval from the graduate program director. For graduate students who want to pursue a career at a teaching institution, instruction can lead to more teaching experience in the Five College Consortium, which directly benefits their career goals.


\appsection{How will this initiative be assessed or measured to determine success?}{300}
There are several ways in which the structure of these workshops will lend itself to quantitative and qualitative assessment:

Since this is a series of workshops, we will be able to track attendance from one session to another. We will use this as an indication of success. We consider high participation (25+ students) across all sessions to be successful. Although the workshops are not strictly cumulative, we consider retention between sessions a sign of success as well, especially if those students go on to teach sessions later.

We intend to use GitHub to manage code and participation. This will allow us to monitor public activities feeds from group members, which give us a sense of how engaged students are with the tasks.

Our initial cohort of instructors will develop materials for teaching the course. As attendees gain proficiency, they will be able to lead workshop sessions using those materials. We will be able to measure the flow from attendee to instructor.

Since these workshops are ongoing, we will also end each session with surveys that ask for demographic information and feedback on the topic, level of detail presented, pace of instruction, and facilities. We will incorporate useful feedback from session to session.

\end{document}